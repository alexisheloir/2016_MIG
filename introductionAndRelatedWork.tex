\section{Introduction}

Since the advent of multitouch enabled smartphones, multi-touch input has become a ubiquitous input modality. Recent development and advances in acquisition devices and computer vision have today the potential to empower users with novel interaction modalities referred as mid-air gestures and depicted in Figure 1. %~\ref{fig:moneyshot}.
If mid-air gestural input is promised to become as widespread as multitouch touch gestural input, it is necessary to conduct the ergonomic studies and to provide a set of relevant design guidelines which will guarantee a safe and optimal user experience.

Even if the literature in the field of biomechanics states that a prolonged solicitation of certain muscles and joints leads to chronic pain and even functional disability \cite{franklyn-miller_biomechanical_2014}, the impact of an extended usage of mid-air gestural input on the musculoskeletal system is yet to be known in a principled manner.  In the field of Human Computer Interaction, it has been observed that a prolonged interaction with a mid-air gesture interface might lead to a feeling of heaviness in the upper limb casually referred as \emph{gorilla-arm} \cite{aigner_understanding_2012}. If many studies have already assessed both the ergonomics and the performance of mid-air gesture \cite{aigner_understanding_2012,hincapie-ramos_consumed_2014,bachynskyi_performance_2015,sridhar_investigating_2015,straker_comparison_2008}, only a few design guidelines have been suggested. Ramoe et al. suggested to limit the amount of time spent by the user while raising her/his arm \cite{hincapie-ramos_consumed_2014}. Aigner et al. highlighted the influence of pointing as well as iconic bimanual gestures \cite{aigner_understanding_2012} on both usability and performance. Kim et al. analyzed smartphone interaction and found that body posture affects the range of motion and muscle activity of the thumb \cite{kim_biomechanical_2012}. They conclude that deeper analyses of ergonomics should be conducted in future work and do not provide any insight on how an interface designer could persuade users to adopt correct postures. We could not, however, find any guideline specifically addressing the interplay between interaction design and ergonomics. In this paper, we introduce and assess the following principle:~ \emph{``encouraging the user to deploy a large variety of muscles during the interaction''}. Our hypothesis is the following: ``A subject who often changes his upper body posture exhibit a broader muscle recruitment. When more muscles are involved in the interaction, each muscle undergoes less strain''. Applying this principle to mid-air gesture interaction should result in an input method with a well-balanced distribution of the muscular load, leading to less muscular strain and higher comfort.

While this guideline suits both touch and mid-air gestural input, it is currently assessed in the particular context of mid-air gesture interaction where the task consists of docking 3D objects. During the task, a persuasive interaction design leveraging both head tracking and parallax shift persuades the user to move his head and therefore change his upper body posture according to our guideline. Using a similar methodology as the one presented in \cite{bachynskyi_performance_2015}, we use optical motion capture, force plate sensors and biomechanical simulation to compute several indices of ergonomics. The use of biomechanical simulation not only provides a rich description of biomechanical events in the body, but it also overcomes the challenges posed by traditional ergonomics instruments such as goniometers and surface EMG.
To assess our hypothesis we perform a 40-participant comparative user study for a 3D docking task to compare pure mid-air gestural interface against mid-air gestural interface with enforced upper-body movements.
During the experiment, we record optical motion capture, which serves as input for biomechanical simulations computing a variety of indices from inside the body, including joint moments, muscle forces and recruitment. Our contribution consists of showing how introducing additional variability of the whole upper body posture has the potential to reduce muscular fatigue and the resulting ``Gorilla arm'' effect during mid-air interaction. We also we introduce and evaluate a persuasive interaction design leveraging head-tracking and parallax effect encouraging postural changes during the interaction.
