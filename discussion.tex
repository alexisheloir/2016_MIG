\section{Discussion and conclusion}

\todo[inline]{Report even if the subjects performance did not change and muscular pain is perceived the same, since users in HT condition were indeed moving more the head their muscular effort in better distributed. This can be an advantage because pain is often perceived only the day after. - FN}

\todo[inline]{Speculations. The system can be used to analyse users behavior. Now offline, in future between gaming sessions or even in real-time. The data on the muscular effrt can be used in future sessions or in real time to fine tune the game parameters or suggest pauses. Of course the methodology can be used already now for design and testing. }

The study and the simulation results support the proposed ``postural variability'' principle for mid-air interaction.
The results show that posture variability decreases the recruitment of the muscles which are traditionally most loaded in mid-air interaction by 15-25\%. 
Surprisingly, the load on the neighbouring and postural muscles is in most, but very few cases also decreased, and the total load on all muscles is decreased as well.
Thus increasing postural variability decreases loads on the whole body and as a result should reduce fatigue and strain. 
This can be explained by the fact that in static posture the muscles have to be activated all the time and often agonists and antagonists are coactivated, while with variable posture the coactivation can be smaller.

The analysis of the Borg RPE self-assessment for pain conducted on the questionnaires revealed no significant differences between the two conditions (with or without head tracking) after 15 and 30 minutes of mid-air gesturing. This suggests that the experiments we conducted might not have had a sufficient duration (30 minutes) to exhibit a significant difference.
Our findings are currently limited to the single type of mid-air task, and to better generalize the ``postural variability'' principle future work has to test it on other types of mid-air tasks as well as on the tasks with touch screen interactions, or handheld devices.
The analysis conducted for this paper focused on the upper body. Future work will be conducted on the whole body in order to better understand the role of the hips and the lower limbs in the adaptation of the posture while working in sitting positions.
